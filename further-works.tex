
\section{Further Works}

To improve the framework, here are some suggestions for the next steps. As it mentioned before, we can try to define the score directly according to the t value. So some different function will be tested to find an acceptable funcion.
Combination of amplitude and rate score can be done by simple average or by assigning weigh to each of the scores corrisponding to its importance. So that is something to work on.
Huge amount of data is available from the simulation domain area and we want to do the validation using all of those data. Similar procedure can be use with some modifications.The  comparisopns will be the same, except that it might be neceessary to select two different approximated lines for two different segments of the data and synthetics to get a more realistic approximation and more accurate results. In that case, the scores will be calculated for each segment separetely and then they will be combined with a simple or weighted average. In addition, with having many synthetic points, the amplitude score, instead of for each station, can be calculated in selected steps for the approximated line. The next step would be to check the results of synthetics with NGA predictions.


