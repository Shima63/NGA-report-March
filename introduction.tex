\section{Introduction}

Increasing advances in ground motion simulation emboldens researches to seek for reliable methods of simulation validation. Validation is an effort to check the correctness of the formulation and the accuracy with respect to observations. The concept of simulation has found its way from systems science  (e.g. Sargent, 2005; Oberkampf et al., 2004) to earthquake simulation community (e.g., Bielak et al.,2010; Chaljub et al., 2010). Up untill now, validaton of ground motion simulations is done by comparing synthetic seismograms against observations from past earthquakes (Olsen and Mayhew, 2010; Taborda and Bielak, 2013 and 2014, Taborda et al., 2016) predominantly using quantitative methods known as goodness-of-fit (GOF) or misfit criteria (\citet{Anderson_2004_Proc}; Olsen and Mayhew, 2010; Kristekova et al., 2009). These validations are concentrating comparisions to the availability of data, which for most of the past events is limited to a reduced number of locations.\par
However, the simulations are resulved for the whole computational domain, not just the specific stations, and this provide us with a complete surface datsets which can be used to build simulation specific attenuation curves comparable to well-established attenuation relationships.The goal of this project is to shape a new simulation validition framework based on comparision with attenuation relationships. To this end, we start with the available recorded data for southern california region from past earthquakes and synthetics from four different velocity model to define and calibrate a goodness of fit criteria using their corrisponding attenuation relationships. Later we will extend the findings to the whole simulation domain and will perform the evaluation between NGA-West2 prediction and our physic-based synthetics.  
